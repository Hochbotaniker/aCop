\documentclass[12pt,a4paper]{article}
\usepackage[ngerman]{babel}
\usepackage{amsmath}
\usepackage{amssymb}
\usepackage{listings}
\usepackage{ifsym}
\setlength{\parindent}{0pt}
\author{Steffen Anhäuser, Jan-Erik Menzel, Melina Morch, Lukas ?}
\title{ACOP UE 01}
\date{}

\begin{document}
\maketitle

\section*{Aufgabe 1.1}

\subsection*{a)}
Definieren Sie die Terme der abstrakten Grammatik analog zur Vorlesung:
\subsubsection*{Rekursiv:}
$ 1. \{0\} \subseteq T $
\\
2. if $t_1 \in T$ 
then 
\{succ  $t_1$, pred  $t_1$, move $t_1$, turn $t_1\} \subseteq T$.
\\
3. if $t_1 \in T, t_2 \in T, t_3 \in T, t_4 \in T, t_p \in T, t_o \in T$ then \{board $t_1 \, t_2 \, t_3 \, t_4 \, t_p \, t_o \}~\subseteq~T$.
\\
\subsubsection*{Mit Inferenzregeln:}
$ 0 \in T  
\\
\frac{t_1 \in T}{\text{succ }t_1 \in T} \frac{t_1 \in T}{\text{pred }t_1 \in T} \frac{t_1 \in T}{\text{move }t_1 \in T} \frac{t_1 \in T}{\text{turn }t_1 \in T}
\\
\frac{t_1 \in T, t_2 \in T, t_3 \in T, t_4 \in T, t_p \in T, t_o \in T}{\text{board }t_1 \, t_2 \, t_3 \, t_4 \, t_p \, t_o t_1 \in T}
$ 
\\
\subsubsection*{Konkret:}
$S_0 = \varnothing  
\\
S_{i+1} = \{0\} \cup \{ \text{succ } t_1, \text{pred } t_1, \text{move } t_1, \text{turn } t_1 |  \;t_1 \ \text{in } S_i\} 
\\ \cup \{\text{board } t_1, t_2, t_3, t_4, t_p, t_o | \; t_1, t_2, t_3, t_4, t_p, t_o \text{ in } S_i\}
\\
S = \cup_i \; S_i$

\subsection*{b)}
Welche Terme enthält die Menge $S_2$?
\\
$S_2$ enthält die Konstante und jede Kombination von einer Konstante mit den Termen pred, succ, move, turn, board.
\\
$ S_2 = \{0,\text{pred}(0), \text{move}(0), \text{turn}(0), \text{succ}(0),\text{board}(0,0,0,0,0,0)\}$
\subsection*{c)}
Stellen Sie eine Formel zur Berechnung der Mächtigkeit der kumulativen Mengen $S_i$ auf.
Wie viele Elemente enthält die Menge $S_3$?
\\
$ |S_{i+1}|=|S_i|^6 + |S_i| \times 4 + 1$ and $|S_0| = 0
\\|S_0|=0 \: |S_1|=1 \: |S_2|=6
\\|S_3| = 46681 $

\section*{Aufgabe 1.2}

\subsection*{a)}
Stellen Sie analog zur Vorlesung ein System von Evaluationsregeln für G1 auf.

$1. \:\text{pred}(\text{succ}(t)) \rightarrow t 
\\
2. \:\text{succ}(\text{pred}(t)) \rightarrow t 
\\
3. \: \text{turn}(\text{turn}(\text{turn}(\text{turn}(t)))) \rightarrow t
\\
4. \: \text{turn}(t) \rightarrow succ(t)
$
\subsection*{b)}

\section*{Aufgabe 1.3}
Betrachten Sie die abstrakte Grammatik aus der Vorlesung. Angenommen, die Evaluationsstrategie soll so angepasst werden, dass die then und else Fälle einer if-Anweisung (in genau dieser Reihenfolge) vor dem Guard ausgewertet werden sollen. Ändern Sie die Evaluationsregeln so ab, dass diese Strategie umgesetzt wird.\footnote{Verstehen wir das richtig, wenn mit Guard die verkürzte Auswertung gemeint ist und damit nicht mehr z.B. t2 ausgewertet werden muss, sondern nur noch t1?}

$1. \: t_2 \: t_3$ if true then true else false $\rightarrow t_2 $
\\
$2. \: t_2 \: t_3$ if false then false else true $\rightarrow t_3 $
\\
Gibt es bei der dritten Regel überhaupt einen Guard? Dort findet ja keine verkürzte Auswertung statt.
\end{document}


